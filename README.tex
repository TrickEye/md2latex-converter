\documentclass{ctexart}
\usepackage{graphicx}
\usepackage{hyperref}
\title{\texttt{md2latex-converter} introduction }
\begin{document}
	\maketitle

	> \texttt{pip install md2latex-converter}

	\texttt{md2latex-converter} is a Python package that helps to convert a .md (Markdown) file into a .tex (LaTeX) file, with special support for Chinese, using the \texttt{ctex} package provided by the LaTeX community.

	\texttt{md2latex-converter} 是一个将 Markdown 文件转换成 LaTeX 源代码的 Python 工具包。使用了 LaTeX 的 \texttt{ctex} 包,因此对于 中文的编码环境有特别的关照。

	\section{Installation and Usage | 安装与使用}

	\texttt{pip install md2latex-converter}

	This will install the package into your current python interpreter. The installation will add \texttt{m2l.exe} into the \texttt{Scripts} folder of python (on Windows) or \texttt{m2l} into \texttt{~/.local/bin/} by default (on Linux), if you have previously added the path above into your system PATH variable, you should be able to invoke the program through the command \texttt{m2l}.

	这会在现有的 Python 解释器中安装此包。安装过程中会将 \texttt{m2l.exe} 安装在 当前的 Python 的 \texttt{Scripts} 路径(Windows)或默认将 \texttt{m2l} 安装到 \texttt{~/.local/bin/} 路径(Linux)。如果此前已经将这个路径加入了 PATH 变量, 那么可以通过 \texttt{m2l} 指令来运行此程序。

	\texttt{m2l file.md}

	This will read and convert the content in \texttt{file.md} into \texttt{file.tex} at the current working directory. After the conversion, use \texttt{xelatex file} to produce a \texttt{.pdf} file from the LaTeX source code.

	The output filename depends on your input, for \texttt{foo.md}, m2l will produce \texttt{foo.tex}

	这会读取并转换 \texttt{file.md} 的内容到当前工作路径的 \texttt{file.tex} 文件。在此之后,可以使用 \texttt{xelatex file} 来编译产生 pdf

	输出的文件名由命令输入决定,对 \texttt{foo.md} 的转换会产生 \texttt{foo.tex}

	\section{Current progress and plans | 进度,安排}

	\begin{itemize}
		\item Currently \texttt{m2l} basically supports:
		\begin{itemize}
			\item plain text,
			\item title,
			\item unordered/ordered lists,
			\item pictures (please use a local path if you do so, otherwise you are being impolite to LaTeX.)
			\item inline patterns
			\begin{itemize}
				\item something \textbf{bold}
				\item something \textit{italic}
				\item or something \textbf{\textit{bold and italic}}
				\item inline \texttt{code snippets}
				\item \href{https://http.cat/404}{hyperlinks}
			\end{itemize}
		\end{itemize}
		\item \texttt{m2l} does not intend to support:
		\begin{itemize}
			\item tables, as writing tables in markdown is not a very swift thing to do.
			\item html labels, as the target file format is \texttt{.tex} or \texttt{.pdf}, which is not compatible with online stuff.
		\end{itemize}
		\item The future versions will focus on equations, codeblocks
		\item Versions in the more distant future will support DIY markdown grammar and texify methods.
	\end{itemize}

	\begin{itemize}
		\item 现阶段支持了:
		\begin{itemize}
			\item 文本
			\item 标题
			\item 有序无序列表
			\item 图片(本地路径)
			\item 行内样式
			\begin{itemize}
				\item \textbf{粗体}文本
				\item *斜体*文本
				\item \textbf{\textit{又粗又斜}} 的文本(你为什么要这样干)
				\item \texttt{代码}片段
				\item \href{https://http.cat/404}{超链接}
			\end{itemize}
		\end{itemize}
		\item 这个包不计划支持的东西包括
		\begin{itemize}
			\item 表格,因为用markdown写表格不是很方便
			\item html标签,因为目标代码是 \texttt{.tex} 或者 \texttt{.pdf} ,这不是一个很适配在线内容的格式。
		\end{itemize}
		\item 未来版本计划支持公式、代码块
		\item 在更久远的未来,可以支持用户自定义md语法和 texify 方法
	\end{itemize}

\end{document}
